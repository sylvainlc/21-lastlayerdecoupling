\documentclass{article}
% Packages
\usepackage{amsmath} % Math environments
\usepackage{amsfonts, dsfont} % Math fonts
\usepackage{authblk} % Author and affiliations
\usepackage{hyperref} % Links
\usepackage{graphicx} % Include graphics
\usepackage{apalike} % Reference styling

% Commands
\providecommand{\keywords}[1]{\textbf{\textit{Index terms---}} #1}

\title{Rebuttal for Reviewer 2163}

\author{Max Cohen}
\affil{Samovar, T\'el\'ecom SudParis, CITI, TIPIC, Institut Polyechnique de Paris}
\date{}

\begin{document}
\maketitle

\begin{itemize}
	\item Lancer un training sur une base de donnees publique ; on a choisi https://archive.ics.uci.edu/ml/datasets/Appliances+energy+prediction.
	\item Ajouter une table aves la comparison de notre modèle avec le LSTM dropout pour les metriques \textbf{PICP, MPIW, Temps d'inférence}.
	\item Reformuler les motivations pour expliquer clairement l'intérêt des modèles d'état
	\item Expliquer qu'on peut s'adapter à n'importe quel bruit en changeant les équations
	\item Expliquer que le temps de calcul pour l'entraînement est raisonable, et qu'on prévoit de le réduire en appliquant des méthodes prouvées.
\end{itemize}

\bibliographystyle{apalike}
\bibliography{references}
\end{document}
