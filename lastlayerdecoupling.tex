% Template for ICASSP-2021 paper; to be used with:
%          spconf.sty  - ICASSP/ICIP LaTeX style file, and
%          IEEEbib.bst - IEEE bibliography style file.
% --------------------------------------------------------------------------
\documentclass{article}
\usepackage{spconf,amsmath,graphicx}

\title{Last layer state space model}
%
% Single address.
% ---------------
\name{Author(s) Name(s)\thanks{Thanks to XYZ agency for funding.}}
\address{Author Affiliation(s)}
%
% For example:
% ------------
%\address{School\\
%	Department\\
%	Address}
%
% Two addresses (uncomment and modify for two-address case).
% ----------------------------------------------------------
%\twoauthors
%  {A. Author-one, B. Author-two\sthanks{Thanks to XYZ agency for funding.}}
%	{School A-B\\
%	Department A-B\\
%	Address A-B}
%  {C. Author-three, D. Author-four\sthanks{The fourth author performed the work
%	while at ...}}
%	{School C-D\\
%	Department C-D\\
%	Address C-D}
%

\begin{document}
\maketitle

\begin{abstract}
	As sequential neural architectures become deeper and more complex, estimating the uncertainty of their predictions is a very challenging problem.
	Efforts in quantifying uncertainty often rely on specific training procedures, and bear additional computational costs due to the dimensionality of such models.
	In this paper, we demonstrate how uncertainty can be estimated on top of an existing and trained neural network, combining a state space-based last  layer and  Sequential Monte Carlo methods. This approach allows to separate representation learning and uncertainty quantification. We apply our proposed methodology for the estimation of air quality in office buildings, through the hourly prediction of \ensuremath{\mathrm{CO_2}} levels.
	Our model accounts for the noisy data structure, due to unknown or unavailable variables (occupancy of the building, manual ventilation, etc.), and is able to provide confidence intervals on \ensuremath{\mathrm{CO_2}} predictions.
	We believe a deeper understanding of \ensuremath{\mathrm{CO_2}} variations can assist in regulating and reducing HVAC consumption, while improving well being.
\end{abstract}

\begin{keywords}
	One, two, three, four, five
\end{keywords}

\section{Introduction}
\label{sec:intro}
\begin{itemize}
	\item Describe the active research on bayesian RNN and uncertainty for time series (cf Alice paper with transformer).
	\item Describe the specific application of predicting air quality, and consumptions with high social impact (use Oze Energies information).
	\item Describe the aim: separating  representation learning and uncertainty estimation (cf Nicolas Brosse paper on last layer classification).
	\item Describe your protocol: a new state space model on the last layer, trained with SMC.
\end{itemize}


\section{Background}
\label{sec:background}

\subsection{Recurrent neural networks}
\label{sec:background:rnn}
\begin{itemize}
	\item Short background on RNN, GRU, LSTM (and maybe transformer as it could be extended to those models).
	\item Detail the equations for at least the generic model implemented later in the paper.
\end{itemize}

\subsection{Sequential Monte Carlo methods}
\label{sec:background:smc}
\begin{itemize}
	\item Provide a generic introduction to SMC.
	\item Explain the use of SMC to compute gradients with Fisher's identity (gradient descent and/or EM).
\end{itemize}

\section{Last layer decoupling}
\label{sec:decoupling}

\begin{itemize}
	\item Detail the protocol : training / last layer noise / SMC.
	\item Detail the applications: fine tuning of last layer weights, short-time prediction with confidence intervals.
\end{itemize}

\section{Experiments}
\label{sec:exp}
\begin{itemize}
	\item Smoothed particles trajectories' associated prediction: sample particles from the posterior, smooth the trajectories, apply the model function, and plot the results as a interval.
	\item prediction t+1: plot t+1 predictions as boxplots, along with the mean and the observations.
	\item Averaged MSE: sample particles for half a week. For the second half, observations are not available at all ; thus we either take the mean of the predictions associated with each particles, or sample from the observation model and take the mean.
\end{itemize}

\subsection{Simulated data}
\label{sec:exp:synthetic}

\subsection{Air quality}
\begin{itemize}
	\item train kalman filter, sample under the gaussian law, plot boxplot
	\item compare LSTM MC dropout, same boxplot (should be smaller)
\end{itemize}
\label{sec:exp:airquality}

% Below is an example of how to insert images. Delete the ``\vspace'' line,
% uncomment the preceding line ``\centerline...'' and replace ``imageX.ps''
% with a suitable PostScript file name.
% -------------------------------------------------------------------------
%\begin{figure}[htb]

%\begin{minipage}[b]{1.0\linewidth}
%  \centering
%  \centerline{\includegraphics[width=8.5cm]{image1}}
%%  \vspace{2.0cm}
%  \centerline{(a) Result 1}\medskip
%\end{minipage}
%%
%\begin{minipage}[b]{.48\linewidth}
%  \centering
%  \centerline{\includegraphics[width=4.0cm]{image3}}
%%  \vspace{1.5cm}
%  \centerline{(b) Results 3}\medskip
%\end{minipage}
%\hfill
%\begin{minipage}[b]{0.48\linewidth}
%  \centering
%  \centerline{\includegraphics[width=4.0cm]{image4}}
%%  \vspace{1.5cm}
%  \centerline{(c) Result 4}\medskip
%\end{minipage}
%%
%\caption{Example of placing a figure with experimental results.}
%\label{fig:res}
%%
%\end{figure}


% To start a new column (but not a new page) and help balance the last-page
% column length use \vfill\pagebreak.
% -------------------------------------------------------------------------
%\vfill
%\pagebreak

\section{COPYRIGHT FORMS}
\label{sec:copyright}

You must submit your fully completed, signed IEEE electronic copyright release
form when you submit your paper. We {\bf must} have this form before your paper
can be published in the proceedings.

\section{RELATION TO PRIOR WORK}
\label{sec:prior}

The text of the paper should contain discussions on how the paper's
contributions are related to prior work in the field. It is important
to put new work in  context, to give credit to foundational work, and
to provide details associated with the previous work that have appeared
in the literature. This discussion may be a separate, numbered section
or it may appear elsewhere in the body of the manuscript, but it must
be present.

You should differentiate what is new and how your work expands on
or takes a different path from the prior studies. An example might
read something to the effect: "The work presented here has focused
on the formulation of the ABC algorithm, which takes advantage of
non-uniform time-frequency domain analysis of data. The work by
Smith and Cohen \cite{Lamp86} considers only fixed time-domain analysis and
the work by Jones et al \cite{C2} takes a different approach based on
fixed frequency partitioning. While the present study is related
to recent approaches in time-frequency analysis [3-5], it capitalizes
on a new feature space, which was not considered in these earlier
studies."

\vfill\pagebreak

\section{REFERENCES}
\label{sec:refs}

List and number all bibliographical references at the end of the
paper. The references can be numbered in alphabetic order or in
order of appearance in the document. When referring to them in
the text, type the corresponding reference number in square
brackets as shown at the end of this sentence \cite{C2}. An
additional final page (the fifth page, in most cases) is
allowed, but must contain only references to the prior
literature.

% References should be produced using the bibtex program from suitable
% BiBTeX files (here: strings, refs, manuals). The IEEEbib.bst bibliography
% style file from IEEE produces unsorted bibliography list.
% -------------------------------------------------------------------------
\bibliographystyle{IEEEbib}
\bibliography{refs.bib}

\end{document}
