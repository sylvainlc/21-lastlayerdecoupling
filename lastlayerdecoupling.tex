% Template for ICASSP-2021 paper; to be used with:
%          spconf.sty  - ICASSP/ICIP LaTeX style file, and
%          IEEEbib.bst - IEEE bibliography style file.
% --------------------------------------------------------------------------
\documentclass{article}
\usepackage{spconf,amsmath,graphicx}

\title{Last layer state space model}
%
% Single address.
% ---------------
\name{Author(s) Name(s)\thanks{Thanks to XYZ agency for funding.}}
\address{Author Affiliation(s)}
%
% For example:
% ------------
%\address{School\\
%	Department\\
%	Address}
%
% Two addresses (uncomment and modify for two-address case).
% ----------------------------------------------------------
%\twoauthors
%  {A. Author-one, B. Author-two\sthanks{Thanks to XYZ agency for funding.}}
%	{School A-B\\
%	Department A-B\\
%	Address A-B}
%  {C. Author-three, D. Author-four\sthanks{The fourth author performed the work
%	while at ...}}
%	{School C-D\\
%	Department C-D\\
%	Address C-D}
%

\begin{document}
\maketitle

\begin{abstract}
	As sequential neural architectures become deeper and more complex, estimating the uncertainty of their predictions is ever so challenging.
	Efforts in quantifying uncertainty often rely on specific training procedures, and bear additional computational costs due to the dimensionality of such models.
	In this paper, we demonstrate how uncertainty can be estimated on top of an existing and trained neural network, combining a state space-based last  layer and  Sequential Monte Carlo methods. This approach allows to separate representation learning and uncertainty quantification. We apply our proposed methodology for the estimation of air quality in office buildings, through the hourly prediction of \ensuremath{\mathrm{CO_2}} levels.
	Our model accounts for the noisy data structure, due to unknown or unavailable variables (occupancy of the building, manual ventilation, etc.), and is able to provide confidence intervals on \ensuremath{\mathrm{CO_2}} predictions.
	We believe a deeper understanding of \ensuremath{\mathrm{CO_2}} variations can assist in regulating and reducing HVAC consumption, while improving well being.
\end{abstract}

\begin{keywords}
	One, two, three, four, five
\end{keywords}

\section{Introduction}
\label{sec:intro}
\begin{itemize}
	\item Describe the active research on bayesian RNN and uncertainty for time series (cf Alice paper with transformer).
	\item Describe the aim: separating  representation learning and uncertainty estimation (cf Nicolas Brosse paper on last layer classification).
	\item Describe your protocol: a new state space model on the last layer, trained with SMC.
\end{itemize}

\section{Background}
\label{sec:background}

\subsection{Recurrent neural networks}
\label{sec:background:rnn}
\begin{itemize}
	\item Short background on RNN, GRU, LSTM (and maybe transformer as it could be extended to those models).
	\item Detail the equations for the generic model implemented later in the paper.
\end{itemize}

\subsection{Sequential Monte Carlo methods}
\label{sec:background:smc}
\begin{itemize}
	\item Provide a generic introduction to SMC.
	\item Explain the use of SMC to compute gradients with Fisher's identity (gradient descent and EM).
\end{itemize}

\section{Last layer decoupling}
\label{sec:decoupling}

Provide simple equations for the smc model, along with formulas for the $\Sigma_{x, y}$ updates.

Protocol:
\begin{enumerate}
	\item Training with classic methods (train dataset)
	\item Finetuning with MC (train dataset)
\end{enumerate}

\section{Experiments}
\label{sec:exp}

\subsection{Training}%
\label{sub:training}

\begin{itemize}
	\item Compare EM and gradient training for $\Sigma_{x, y}$ estimation
\end{itemize}

\subsection{Visualizations}%
\label{sub:visualizations}

Visualizations listed in priority order
\begin{enumerate}
	\item Prediction at t+1: plot t+1 predictions as boxplots, along with the mean and the observations.
	\item Prediction at t+k
	\item Compare weights evolution with classic finetuning
	\item Smoothed predictions: Smoothed particles trajectories' associated prediction: sample particles from the posterior, smooth the trajectories, apply the model function, and plot the results as a interval.
\end{enumerate}

\subsection{Evaluations}%
\label{sub:evaluations}

Evaluations listed in priority order
\begin{enumerate}
	\item Compare confidence interval with MC-dropout model
	\item Compare MSE (average on particles) with classic finetuning: sample particles for half a week. For the second half, observations are not available at all ; thus we either take the mean of the predictions associated with each particles, or sample from the observation model and take the mean.
		\item Compare linear SMC with kalman filter, sample under the gaussian law, plot boxplot
\end{enumerate}


% Below is an example of how to insert images. Delete the ``\vspace'' line,
% uncomment the preceding line ``\centerline...'' and replace ``imageX.ps''
% with a suitable PostScript file name.
% -------------------------------------------------------------------------
%\begin{figure}[htb]

%\begin{minipage}[b]{1.0\linewidth}
%  \centering
%  \centerline{\includegraphics[width=8.5cm]{image1}}
%%  \vspace{2.0cm}
%  \centerline{(a) Result 1}\medskip
%\end{minipage}
%%
%\begin{minipage}[b]{.48\linewidth}
%  \centering
%  \centerline{\includegraphics[width=4.0cm]{image3}}
%%  \vspace{1.5cm}
%  \centerline{(b) Results 3}\medskip
%\end{minipage}
%\hfill
%\begin{minipage}[b]{0.48\linewidth}
%  \centering
%  \centerline{\includegraphics[width=4.0cm]{image4}}
%%  \vspace{1.5cm}
%  \centerline{(c) Result 4}\medskip
%\end{minipage}
%%
%\caption{Example of placing a figure with experimental results.}
%\label{fig:res}
%%
%\end{figure}


% To start a new column (but not a new page) and help balance the last-page
% column length use \vfill\pagebreak.
% -------------------------------------------------------------------------
%\vfill
%\pagebreak

\section{REFERENCES}
\label{sec:refs}

% References should be produced using the bibtex program from suitable
% BiBTeX files (here: strings, refs, manuals). The IEEEbib.bst bibliography
% style file from IEEE produces unsorted bibliography list.
% -------------------------------------------------------------------------
\bibliographystyle{IEEEbib}
\bibliography{refs.bib}

\end{document}
