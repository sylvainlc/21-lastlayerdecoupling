\def\year{2022}\relax
%File: formatting-instructions-latex-2022.tex
%release 2022.1
\documentclass[letterpaper]{article} % DO NOT CHANGE THIS
\usepackage{aaai22}  % DO NOT CHANGE THIS
\usepackage{times}  % DO NOT CHANGE THIS
\usepackage{helvet}  % DO NOT CHANGE THIS
\usepackage{courier}  % DO NOT CHANGE THIS
\usepackage[hyphens]{url}  % DO NOT CHANGE THIS
\usepackage{graphicx} % DO NOT CHANGE THIS
\urlstyle{rm} % DO NOT CHANGE THIS
\def\UrlFont{\rm}  % DO NOT CHANGE THIS
\usepackage{natbib}  % DO NOT CHANGE THIS AND DO NOT ADD ANY OPTIONS TO IT
\usepackage{caption} % DO NOT CHANGE THIS AND DO NOT ADD ANY OPTIONS TO IT
\DeclareCaptionStyle{ruled}{labelfont=normalfont,labelsep=colon,strut=off} % DO NOT CHANGE THIS
\frenchspacing  % DO NOT CHANGE THIS
\setlength{\pdfpagewidth}{8.5in}  % DO NOT CHANGE THIS
\setlength{\pdfpageheight}{11in}  % DO NOT CHANGE THIS
%
% These are recommended to typeset algorithms but not required. See the subsubsection on algorithms. Remove them if you don't have algorithms in your paper.
\usepackage{algorithm}
\usepackage{algorithmic}

%
% These are are recommended to typeset listings but not required. See the subsubsection on listing. Remove this block if you don't have listings in your paper.
\usepackage{newfloat}
\usepackage{listings}
\lstset{%
	basicstyle={\footnotesize\ttfamily},% footnotesize acceptable for monospace
	numbers=left,numberstyle=\footnotesize,xleftmargin=2em,% show line numbers, remove this entire line if you don't want the numbers.
	aboveskip=0pt,belowskip=0pt,%
	showstringspaces=false,tabsize=2,breaklines=true}
\floatstyle{ruled}
\newfloat{listing}{tb}{lst}{}
\floatname{listing}{Listing}
%
%\nocopyright
%
% PDF Info Is REQUIRED.
% For /Title, write your title in Mixed Case.
% Don't use accents or commands. Retain the parentheses.
% For /Author, add all authors within the parentheses,
% separated by commas. No accents, special characters
% or commands are allowed.
% Keep the /TemplateVersion tag as is
%\pdfinfo{
%/Title (AAAI Press Formatting Instructions for Authors Using LaTeX -- A Guide)
%/Author (AAAI Press Staff, Pater Patel Schneider, Sunil Issar, J. Scott Penberthy, George Ferguson, Hans Guesgen, Francisco Cruz, Marc Pujol-Gonzalez)
%/TemplateVersion (2022.1)
%}

% DISALLOWED PACKAGES
% \usepackage{authblk} -- This package is specifically forbidden
% \usepackage{balance} -- This package is specifically forbidden
% \usepackage{color (if used in text)
% \usepackage{CJK} -- This package is specifically forbidden
% \usepackage{float} -- This package is specifically forbidden
% \usepackage{flushend} -- This package is specifically forbidden
% \usepackage{fontenc} -- This package is specifically forbidden
% \usepackage{fullpage} -- This package is specifically forbidden
% \usepackage{geometry} -- This package is specifically forbidden
% \usepackage{grffile} -- This package is specifically forbidden
% \usepackage{hyperref} -- This package is specifically forbidden
% \usepackage{navigator} -- This package is specifically forbidden
% (or any other package that embeds links such as navigator or hyperref)
% \indentfirst} -- This package is specifically forbidden
% \layout} -- This package is specifically forbidden
% \multicol} -- This package is specifically forbidden
% \nameref} -- This package is specifically forbidden
% \usepackage{savetrees} -- This package is specifically forbidden
% \usepackage{setspace} -- This package is specifically forbidden
% \usepackage{stfloats} -- This package is specifically forbidden
% \usepackage{tabu} -- This package is specifically forbidden
% \usepackage{titlesec} -- This package is specifically forbidden
% \usepackage{tocbibind} -- This package is specifically forbidden
% \usepackage{ulem} -- This package is specifically forbidden
% \usepackage{wrapfig} -- This package is specifically forbidden
% DISALLOWED COMMANDS
% \nocopyright -- Your paper will not be published if you use this command
% \addtolength -- This command may not be used
% \balance -- This command may not be used
% \baselinestretch -- Your paper will not be published if you use this command
% \clearpage -- No page breaks of any kind may be used for the final version of your paper
% \columnsep -- This command may not be used
% \newpage -- No page breaks of any kind may be used for the final version of your paper
% \pagebreak -- No page breaks of any kind may be used for the final version of your paperr
% \pagestyle -- This command may not be used
% \tiny -- This is not an acceptable font size.
% \vspace{- -- No negative value may be used in proximity of a caption, figure, table, section, subsection, subsubsection, or reference
% \vskip{- -- No negative value may be used to alter spacing above or below a caption, figure, table, section, subsection, subsubsection, or reference

\setcounter{secnumdepth}{0} %May be changed to 1 or 2 if section numbers are desired.

% The file aaai22.sty is the style file for AAAI Press
% proceedings, working notes, and technical reports.
%

% Title

% Your title must be in mixed case, not sentence case.
% That means all verbs (including short verbs like be, is, using,and go),
% nouns, adverbs, adjectives should be capitalized, including both words in hyphenated terms, while
% articles, conjunctions, and prepositions are lower case unless they
% directly follow a colon or long dash
\title{Last layer state space models}
\author{
    %Authors
    % All authors must be in the same font size and format.
%    Written by AAAI Press Staff\textsuperscript{\rm 1}\thanks{With help from the AAAI Publications Committee.}\\
%    AAAI Style Contributions by Pater Patel Schneider,
%    Sunil Issar,\\
%    J. Scott Penberthy,
%    George Ferguson,
%    Hans Guesgen,
%    Francisco Cruz\equalcontrib,
%    Marc Pujol-Gonzalez\equalcontrib
}
\affiliations{
    %Afiliations
    \textsuperscript{\rm 1}Association for the Advancement of Artificial Intelligence\\
    % If you have multiple authors and multiple affiliations
    % use superscripts in text and roman font to identify them.
    % For example,

    % Sunil Issar, \textsuperscript{\rm 2}
    % J. Scott Penberthy, \textsuperscript{\rm 3}
    % George Ferguson,\textsuperscript{\rm 4}
    % Hans Guesgen, \textsuperscript{\rm 5}.
    % Note that the comma should be placed BEFORE the superscript for optimum readability

    2275 East Bayshore Road, Suite 160\\
    Palo Alto, California 94303\\
    % email address must be in roman text type, not monospace or sans serif
    publications22@aaai.org
%
% See more examples next
}

%Example, Single Author, ->> remove \iffalse,\fi and place them surrounding AAAI title to use it
\iffalse
\title{My Publication Title --- Single Author}
\author {
    Author Name
}
\affiliations{
    Affiliation\\
    Affiliation Line 2\\
    name@example.com
}
\fi

\iffalse
%Example, Multiple Authors, ->> remove \iffalse,\fi and place them surrounding AAAI title to use it
\title{My Publication Title --- Multiple Authors}
\author {
    % Authors
    First Author Name,\textsuperscript{\rm 1}
    Second Author Name, \textsuperscript{\rm 2}
    Third Author Name \textsuperscript{\rm 1}
}
\affiliations {
    % Affiliations
    \textsuperscript{\rm 1} Affiliation 1\\
    \textsuperscript{\rm 2} Affiliation 2\\
    firstAuthor@affiliation1.com, secondAuthor@affilation2.com, thirdAuthor@affiliation1.com
}
\fi


% REMOVE THIS: bibentry
% This is only needed to show inline citations in the guidelines document. You should not need it and can safely delete it.
\usepackage{bibentry}
% END REMOVE bibentry

\begin{document}

\maketitle

\begin{abstract}
    As sequential neural architectures become deeper and more complex, estimating the uncertainty of their predictions is a very challenging problem.
    Efforts in quantifying uncertainty often rely on specific training procedures, and bear additional computational costs due to the dimensionality of such models.
    In this paper, we demonstrate how uncertainty can be estimated on top of an existing and trained neural network, combining a state space-based last  layer and  Sequential Monte Carlo methods. This approach allows to separate representation learning and uncertainty quantification. We apply our proposed methodology for the estimation of air quality in office buildings, through the hourly prediction of \ensuremath{\mathrm{CO_2}} levels.
    Our model accounts for the noisy data structure, due to unknown or unavailable variables (occupancy of the building, manual ventilation, etc.), and is able to provide confidence intervals on \ensuremath{\mathrm{CO_2}} predictions.
    We believe a deeper understanding of \ensuremath{\mathrm{CO_2}} variations can assist in regulating and reducing HVAC consumption, while improving well being.
\end{abstract}

\section{Introduction}
\label{sec:intro}
\begin{itemize}
\item Describe the active research on bayesian RNN and uncertainty for time series (cf Alice paper with transformer).
\item Describe the specific application of predicting air quality, and consumptions with high social impact (use Oze Energies information).
\item Describe the aim: separating  representation learning and uncertainty estimation (cf Nicolas Brosse paper on last layer classification).
\item Describe your protocol: a new state space model on the last layer, trained with SMC.
\end{itemize}


\section{Background}
\label{sec:background}

\subsection{Recurrent neural networks}
\label{sec:background:rnn}
\begin{itemize}
\item Short background on RNN, GRU, LSTM (and maybe transformer as it could be extended to those models).
\item Detail the equations for at least the generic model implemented later in the paper.
\end{itemize}

\subsection{Sequential Monte Carlo methods}
\label{sec:background:smc}
\begin{itemize}
\item Provide a generic introduction to SMC.
\item Explain the use of SMC to compute gradients with Fisher's identity (gradient descent and/or EM).
\end{itemize}

\section{Last layer decoupling}
\label{sec:decoupling}

\begin{itemize}
\item Detail the protocol : training / last layer noise / SMC.
\item Detail the applications: fine tuning of last layer weights, short-time prediction with confidence intervals.
\end{itemize}

\section{Experiments}
\label{sec:exp}
\begin{itemize}
	\item Smoothed particles trajectories' associated prediction:
		sample particles from the posterior, smooth the trajectories, apply the model function, and plot the results as a interval.
	\item prediction t+1: plot t+1 predictions as boxplots, along with the mean and the observations.
	\item Averaged MSE:
		sample particles for half a week. For the second half, observations are not available at all ; thus we either take the mean of the predictions associated with each particles, or sample from the observation model and take the mean.
\end{itemize}

\subsection{Simulated data}
\label{sec:exp:synthetic}

\subsection{Air quality}
\begin{itemize}
	\item train kalman filter, sample under the gaussian law, plot boxplot
	\item compare LSTM MC dropout, same boxplot (should be smaller)
\end{itemize}
\label{sec:exp:airquality}

\section{Conclusion}
\label{sec:conclusion}
\end{document}
